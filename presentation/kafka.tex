\documentclass{beamer}
% Try the class options [notes], [notes=only], [trans], [handout],
% [red], [compress], [draft] and see what happens!

% \usepackage{definitions}
\usepackage[british]{babel}
\usepackage{color, soul}
\usepackage{tikz}

%% tikz tricks
% \tikzset{onslide/.code args={<#1>#2}{%
%   \only<#1>{\pgfkeysalso{#2}} 
% }}

\pdfinfo{
        /Title (cim)
        /Creator (LaTeX)
        /Producer (pdflatex)
        /Author (szerzo)
        /CreationDate (datum)
	/Subject (tema)
}


\mode<article> % only for the article version
{
  \usepackage{fullpage}
  \usepackage{hyperref}
}
\mode<presentation>
{
  \usetheme[left,width=0.65in,height=0.55in]{Kolozsvar}
  \setbeamercovered{transparent}
  \setbeamertemplate{navigation symbols}{}
  \setbeamertemplate{footline}%
     {\vspace*{-1.4em}\hspace*{0.66in}\textbf{\insertframenumber/\inserttotalframenumber}\newline\vspace*{0.4em}}
		\setbeamerfont{block title}{size=\larger} % RELSIZE -- html-sizes 
		\usefonttheme{professionalfonts}
		\setbeamercolor{math text}{fg=green!30!red!30!brown}
		\setbeamercolor{normal text in math text}{parent=math text}
}

\setbeamercovered{dynamic}

% The following info should normally be given in you main file:
\title[Apache Kafka]{Apache Kafka}
%
\author{ Hunor Ördög, Norbert-Raymond Pap}
%
\institute[UBB Cluj-Napoca]{
  Department of Mathematics and Informatics\\
  Babe{\c{s}}--Bolyai University, Cluj-Napoca}
%
\date{2024 May}


\begin{document}

\frame{\maketitle}

\mode<presentation>
{
  % \begin{frame}
  %   \frametitle{Talk structure}
  % \tableofcontents
  % \end{frame} 

  % \AtBeginSection[]
  {
      \begin{frame}<beamer>{Contents}
        % \tableofcontents[currentsection,currentsubsection,hideothersubsections]
        \tableofcontents
      \end{frame}
    }
}

%%%%%%%%%%%%%%%%%%%%%%%%%%%%%%%%%%%%%%%%%%%%%%%%%%%%%%%%%%%%%%%%%%%%%%
\section[What is Kafka?]{What is Kafka?}

\begin{frame}{What is Kafka?}
  \includegraphics[scale=0.25]{fig/kafka_logo.png}
  \vspace*{2em}
  \begin{itemize}
    \item Open-source \underline{distributed event streaming platform}.
  \end{itemize}
\end{frame}

\begin{frame}{What is a “distributed streaming platform”?}
  \begin{itemize}
    \item Streams are just infinite data, data that never ends.
    \item Distributed means Kafka works in a cluster, each node in the cluster is called a \textbf{Broker}.
          % Brokers are just servers executing a copy of apache kafka.
          \vspace*{1em}
    \item Kafka is a set of machines working together to be able to handle and process real-time infinite data.
  \end{itemize}
\end{frame}


\begin{frame}{Where does Kafka come from?}
  \begin{itemize}
    \item Kafka was originally developed at LinkedIn \includegraphics[scale=0.006]{fig/linkedin_logo.png} in 2010
    \item Open sourced in early 2011
  \end{itemize}
  \includegraphics[scale=0.17]{fig/apache_software.png}
\end{frame}
% 2012 ota Apache Kafka neven ismert es azota ok tartsak karban

\section[Kafka components \& Internal Architecture]{Kafka components \& Internal Architecture}

\begin{frame}{Kafka Components}
  \begin{itemize}
    \item Producer
    \item Consumer
    \item Broker
    \item Cluster
    \item Topic
    \item Partitions
    \item Offset
    \item Consumer Groups
    \item Zookeeper
  \end{itemize}
  \begin{tikzpicture}[remember picture, overlay]
    \node[left=3em] at (current page.east)
    {
      \includegraphics[width=0.35\textwidth]{fig/kafka_logo.png}
    };
  \end{tikzpicture}
\end{frame}

\begin{frame}{Kafka Components}
  \begin{itemize}
    \item \hl{Producer}
    \item Consumer
    \item Broker
    \item Cluster
    \item Topic
    \item Partitions
    \item Offset
    \item Consumer Groups
    \item Zookeeper
  \end{itemize}
  \begin{tikzpicture}[remember picture, overlay]
    \node[left=3em] at (current page.east)
    {
      \includegraphics[width=0.35\textwidth]{fig/kafka_logo.png}
    };
  \end{tikzpicture}
\end{frame}

\begin{frame}{Kafka Components}{Producer}
  \begin{itemize}
    \item A producer is the source of data who will publish events.
  \end{itemize}
  \vspace*{1.5em}
  \includegraphics[scale=0.192]{fig/producer.png}
\end{frame}

\begin{frame}{Kafka Components}
  \begin{itemize}
    \item Producer
    \item \hl{Consumer}
    \item Broker
    \item Cluster
    \item Topic
    \item Partitions
    \item Offset
    \item Consumer Groups
    \item Zookeeper
  \end{itemize}
  \begin{tikzpicture}[remember picture, overlay]
    \node[left=3em] at (current page.east)
    {
      \includegraphics[width=0.35\textwidth]{fig/kafka_logo.png}
    };
  \end{tikzpicture}
\end{frame}

\begin{frame}{Kafka Components}{Consumer}
  \begin{itemize}
    \item A consumer acts as a receiver, consumes events.
  \end{itemize}
  \vspace*{1.5em}
  \includegraphics[scale=0.16]{fig/consumer.png}
\end{frame}

\begin{frame}{Kafka Components}
  \begin{itemize}
    \item Producer
    \item Consumer
    \item \hl{Broker}
    \item Cluster
    \item Topic
    \item Partitions
    \item Offset
    \item Consumer Groups
    \item Zookeeper
  \end{itemize}
  \begin{tikzpicture}[remember picture, overlay]
    \node[left=3em] at (current page.east)
    {
      \includegraphics[width=0.35\textwidth]{fig/kafka_logo.png}
    };
  \end{tikzpicture}
\end{frame}

\begin{frame}{Kafka Components}{Broker}
  \begin{itemize}
    \item The Kafka Broker is the Kafka server.
    \item A broker is an intermediate entity that helps in message exchanges between producers and consumers.
  \end{itemize}
  \vspace*{1.5em}
  \includegraphics[scale=0.15]{fig/broker.png}
\end{frame}

\begin{frame}{Kafka Components}
  \begin{itemize}
    \item Producer
    \item Consumer
    \item Broker
    \item \hl{Cluster}
    \item Topic
    \item Partitions
    \item Offset
    \item Consumer Groups
    \item Zookeeper
  \end{itemize}
  \begin{tikzpicture}[remember picture, overlay]
    \node[left=3em] at (current page.east)
    {
      \includegraphics[width=0.35\textwidth]{fig/kafka_logo.png}
    };
  \end{tikzpicture}
\end{frame}

\begin{frame}{Kafka Components}{Cluster}
  \begin{itemize}
    \item There can be one or more brokers in the Kafka cluster.
    \item Its distributed architecture is one of the reasons that made Kafka so popular.
    \item The Brokers is what makes it so resilient, reliable, scalable, and fault-tolerant.
  \end{itemize}
  \vspace*{0.1em}
  \includegraphics[scale=0.15]{fig/cluster.png}
\end{frame}

\begin{frame}{Kafka Components}
  \begin{itemize}
    \item Producer
    \item Consumer
    \item Broker
    \item Cluster
    \item \hl{Topic}
    \item Partitions
    \item Offset
    \item Consumer Groups
    \item Zookeeper
  \end{itemize}
  \begin{tikzpicture}[remember picture, overlay]
    \node[left=3em] at (current page.east)
    {
      \includegraphics[width=0.35\textwidth]{fig/kafka_logo.png}
    };
  \end{tikzpicture}
\end{frame}

\begin{frame}{Kafka Components}{Topic}
  \begin{itemize}
    \item Specifies the category of the message or the classification of the message.
    \item Consumers can then just respond to the messages that belong to the topics they are listening on.
  \end{itemize}
  \includegraphics[width=0.98\textwidth]{fig/topic.png}
\end{frame}

\begin{frame}{Kafka Components}
  \begin{itemize}
    \item Producer
    \item Consumer
    \item Broker
    \item Cluster
    \item Topic
    \item \hl{Partitions}
    \item Offset
    \item Consumer Groups
    \item Zookeeper
  \end{itemize}
  \begin{tikzpicture}[remember picture, overlay]
    \node[left=3em] at (current page.east)
    {
      \includegraphics[width=0.35\textwidth]{fig/kafka_logo.png}
    };
  \end{tikzpicture}
\end{frame}

\begin{frame}{Kafka Components}{Partitions}
  \begin{itemize}
    \item
  \end{itemize}
\end{frame}

\begin{frame}{Kafka Components}
  \begin{itemize}
    \item Producer
    \item Consumer
    \item Broker
    \item Cluster
    \item Topic
    \item Partitions
    \item \hl{Offset}
    \item Consumer Groups
    \item Zookeeper
  \end{itemize}
  \begin{tikzpicture}[remember picture, overlay]
    \node[left=3em] at (current page.east)
    {
      \includegraphics[width=0.35\textwidth]{fig/kafka_logo.png}
    };
  \end{tikzpicture}
\end{frame}

\begin{frame}{Kafka Components}{Offset}
  \begin{itemize}
    \item
  \end{itemize}
\end{frame}

\begin{frame}{Kafka Components}
  \begin{itemize}
    \item Producer
    \item Consumer
    \item Broker
    \item Cluster
    \item Topic
    \item Partitions
    \item Offset
    \item \hl{Consumer Groups}
    \item Zookeeper
  \end{itemize}
  \begin{tikzpicture}[remember picture, overlay]
    \node[left=3em] at (current page.east)
    {
      \includegraphics[width=0.35\textwidth]{fig/kafka_logo.png}
    };
  \end{tikzpicture}
\end{frame}

\begin{frame}{Kafka Components}{Consumer Groups}
  \begin{itemize}
    \item
  \end{itemize}
\end{frame}

\begin{frame}{Kafka Components}
  \begin{itemize}
    \item Producer
    \item Consumer
    \item Broker
    \item Cluster
    \item Topic
    \item Partitions
    \item Offset
    \item Consumer Groups
    \item \hl{Zookeeper}
  \end{itemize}
  \begin{tikzpicture}[remember picture, overlay]
    \node[left=3em] at (current page.east)
    {
      \includegraphics[width=0.35\textwidth]{fig/kafka_logo.png}
    };
  \end{tikzpicture}
\end{frame}

\begin{frame}{Kafka Components}{Zookeeper}
  \begin{itemize}
    \item
  \end{itemize}
\end{frame}

\end{document}
